\section{排版参考文献及引用}\label{sec:ref}

\subsection{使用bibliography排版参考文献}

编写参考文献一章是个很无聊的工作,且工作量不小。学校目前(2022年)使用的仍2005年的国标,即“GB/T 7714—2005 BibTeX Style”

模板用户需要编辑bibliography.bib文件,填写参考文献的各项属性,如title、author、year等,具体请参考\url{https://github.com/CTeX-org/gbt7714-bibtex-style#%E6%96%87%E7%8C%AE%E7%B1%BB%E5%9E%8B}。OVERLEAF网站对bibtex有比较详细的解释,如果您想要了解关于bibliography文件的基本知识,请参考\url{https://www.overleaf.com/learn/latex/Bibliography_management_with_bibtex#The_bibliography_file},其它关于bibtex的问题也可以参考该网站。其实bib文件的生成工作也可以交给zotero等文献管理软件,进一步实现参考文献管理自动化。

bib文件示例:
{
\color{green!50!black}
\begin{lstlisting}[breaklines=true,]
  @online{x1,
    title = {The Not So Short Introduction to LaTeX2e},
    year = {2021},
    author = {Tobias, O and Hubert, P and Irene, H and Elisabeth, S},
    url = {http://tug.ctan.org/info/lshort/english/lshort.pdf},
    urldate = {2021-06-05},
    langid = {english},
  }
  
  @book{x2,
    title = {LaTeX2e 及常用宏包使用指南},
    author = {李平},
    date = {2004},
    publisher = {{清华大学出版社}},
    location = {{北京}},
    langid = {中文;}
  }
\end{lstlisting}
}

示例中的x1、x2为参考文献的标识符,可以随意设定,在正文中使用\verb|\cite{x1}|命令即可实现文献交叉引用。

bib 文件编辑完成后,用户需要依次进行下面四个操作:编译 latex、编译 bibtex、编译 latex、编译 latex。2022 版编者在 Visual Studio Code 中推荐设置 Recipe

{
\color{green!50!black}
\begin{lstlisting}[breaklines=true,]
{
  "name": "XeLaTeX",
  "tools": [
    "xelatex"
  ]
},
{
  "name": "xelatex -> bibtex -> xelatex*2",
  "tools": [
    "xelatex",
    "bibtex",
    "xelatex",
    "xelatex"
  ]
},
\end{lstlisting}
}

第一项用于在没有引文变化的情况下默认编译一次 latex(耗时较短),第二项用于引文有变化的情况下使用(会比较耗时)。

如果想了解原理,请参见\url{https://liam.page/2016/01/23/using-bibtex-to-generate-reference/}。

若参考文献发生变更,或是编译发生错误,用户需要先清除.aux, .bbl, .blg文件后,重新执行以上操作。

学校规定的参考文献排版规范有一点与国标不同:我校要求英文人名仅首字母大写。因此模板更新了bib样式gbt7714-2005-numerical.bst,已实现仅首字母大写、其余字母小写。并且修改了文章标题单词的大小写问题,现在不会将单词首字母改为小写。